% *** Authors should verify (and, if needed, correct) their LaTeX system  ***
% *** with the testflow diagnostic prior to trusting their LaTeX platform ***
% *** with production work. IEEE's font choices can trigger bugs that do  ***
% *** not appear when using other class files.                            ***
% The testflow support page is at:
% http://www.michaelshell.org/tex/testflow/


%%*************************************************************************
%% Legal Notice:
%% This code is offered as-is without any warranty either expressed or
%% implied; without even the implied warranty of MERCHANTABILITY or
%% FITNESS FOR A PARTICULAR PURPOSE!
%% User assumes all risk.
%% In no event shall IEEE or any contributor to this code be liable for
%% any damages or losses, including, but not limited to, incidental,
%% consequential, or any other damages, resulting from the use or misuse
%% of any information contained here.
%%
%% All comments are the opinions of their respective authors and are not
%% necessarily endorsed by the IEEE.
%%
%% This work is distributed under the LaTeX Project Public License (LPPL)
%% ( http://www.latex-project.org/ ) version 1.3, and may be freely used,
%% distributed and modified. A copy of the LPPL, version 1.3, is included
%% in the base LaTeX documentation of all distributions of LaTeX released
%% 2003/12/01 or later.
%% Retain all contribution notices and credits.
%% ** Modified files should be clearly indicated as such, including  **
%% ** renaming them and changing author support contact information. **
%%
%% File list of work: IEEEtran.cls, New_IEEEtran_how-to.pdf, bare_jrnl_new_sample4.tex,
%%*************************************************************************
\PassOptionsToPackage{unicode}{hyperref}
\PassOptionsToPackage{hyphens}{url}
\PassOptionsToPackage{dvipsnames,svgnames,x11names}{xcolor}
% Note that the a4paper option is mainly intended so that authors in
% countries using A4 can easily print to A4 and see how their papers will
% look in print - the typesetting of the document will not typically be
% affected with changes in paper size (but the bottom and side margins will).
% Use the testflow package mentioned above to verify correct handling of
% both paper sizes by the user's LaTeX system.
%
% Also note that the "draftcls" or "draftclsnofoot", not "draft", option
% should be used if it is desired that the figures are to be displayed in
% draft mode.
%
\documentclass[
  journal,
]{IEEEtran}%
% If IEEEtran.cls has not been installed into the LaTeX system files,
% manually specify the path to it like:
% \documentclass[journal]{../sty/IEEEtran}
\usepackage[cmex10]{amsmath}
\usepackage{amssymb}
\usepackage{iftex}
\ifPDFTeX
  \usepackage[T1]{fontenc}
  \usepackage[utf8]{inputenc}
  \usepackage{textcomp} % provide euro and other symbols
\else % if luatex or xetex
  \usepackage{unicode-math} % this also loads fontspec
  \defaultfontfeatures{Scale=MatchLowercase}
  \defaultfontfeatures[\rmfamily]{Ligatures=TeX,Scale=1}
\fi
%\usepackage{lmodern}
\ifPDFTeX\else
\fi
% Use upquote if available, for straight quotes in verbatim environments
\IfFileExists{upquote.sty}{\usepackage{upquote}}{}
\IfFileExists{microtype.sty}{% use microtype if available
  \usepackage[]{microtype}
  \UseMicrotypeSet[protrusion]{basicmath} % disable protrusion for tt fonts
}{}
\makeatletter
\parindent    1.0em
\ifCLASSOPTIONcompsoc
  \parindent    1.5em
\fi
\makeatother
\usepackage{xcolor}
\setlength{\emergencystretch}{3em} % prevent overfull lines

\setcounter{secnumdepth}{5}
% Make \paragraph and \subparagraph free-standing
\ifx\paragraph\undefined\else
  \let\oldparagraph\paragraph
  \renewcommand{\paragraph}[1]{\oldparagraph{#1}\mbox{}}
\fi
\ifx\subparagraph\undefined\else
  \let\oldsubparagraph\subparagraph
  \renewcommand{\subparagraph}[1]{\oldsubparagraph{#1}\mbox{}}
\fi


\providecommand{\tightlist}{%
  \setlength{\itemsep}{0pt}\setlength{\parskip}{0pt}}\usepackage{longtable,booktabs,array}
\usepackage{calc} % for calculating minipage widths
% Correct order of tables after \paragraph or \subparagraph
\usepackage{etoolbox}
\makeatletter
\patchcmd\longtable{\par}{\if@noskipsec\mbox{}\fi\par}{}{}
\makeatother
% Allow footnotes in longtable head/foot
\IfFileExists{footnotehyper.sty}{\usepackage{footnotehyper}}{\usepackage{footnote}}
\makesavenoteenv{longtable}
\usepackage{graphicx}
\makeatletter
\def\maxwidth{\ifdim\Gin@nat@width>\linewidth\linewidth\else\Gin@nat@width\fi}
\def\maxheight{\ifdim\Gin@nat@height>\textheight\textheight\else\Gin@nat@height\fi}
\makeatother
% Scale images if necessary, so that they will not overflow the page
% margins by default, and it is still possible to overwrite the defaults
% using explicit options in \includegraphics[width, height, ...]{}
\setkeys{Gin}{width=\maxwidth,height=\maxheight,keepaspectratio}
% Set default figure placement to htbp
\makeatletter
\def\fps@figure{htbp}
\makeatother

\usepackage{physics}
\usepackage[version=3]{mhchem}
\usepackage{orcidlink}
\usepackage{float}
\floatplacement{table}{htb}
\makeatletter
\makeatother
\makeatletter
\makeatother
\makeatletter
\@ifpackageloaded{caption}{}{\usepackage{caption}}
\AtBeginDocument{%
\ifdefined\contentsname
  \renewcommand*\contentsname{Table of contents}
\else
  \newcommand\contentsname{Table of contents}
\fi
\ifdefined\listfigurename
  \renewcommand*\listfigurename{List of Figures}
\else
  \newcommand\listfigurename{List of Figures}
\fi
\ifdefined\listtablename
  \renewcommand*\listtablename{List of Tables}
\else
  \newcommand\listtablename{List of Tables}
\fi
\ifdefined\figurename
  \renewcommand*\figurename{Fig.}
\else
  \newcommand\figurename{Fig.}
\fi
\ifdefined\tablename
  \renewcommand*\tablename{Table}
\else
  \newcommand\tablename{Table}
\fi
}
\@ifpackageloaded{float}{}{\usepackage{float}}
\floatstyle{ruled}
\@ifundefined{c@chapter}{\newfloat{codelisting}{h}{lop}}{\newfloat{codelisting}{h}{lop}[chapter]}
\floatname{codelisting}{Listing}
\newcommand*\listoflistings{\listof{codelisting}{List of Listings}}
\makeatother
\makeatletter
\@ifpackageloaded{caption}{}{\usepackage{caption}}
\@ifpackageloaded{subcaption}{}{\usepackage{subcaption}}
\makeatother
\makeatletter
\@ifpackageloaded{tcolorbox}{}{\usepackage[skins,breakable]{tcolorbox}}
\makeatother
\makeatletter
\@ifundefined{shadecolor}{\definecolor{shadecolor}{rgb}{.97, .97, .97}}
\makeatother
\makeatletter
\makeatother
\makeatletter
\makeatother
\usepackage[skip=2pt,font=footnotesize]{caption}
%\captionsetup{format=myformat}
\ifLuaTeX
  \usepackage{selnolig}  % disable illegal ligatures
\fi
\IfFileExists{bookmark.sty}{\usepackage{bookmark}}{\usepackage{hyperref}}
\IfFileExists{xurl.sty}{\usepackage{xurl}}{} % add URL line breaks if available
\urlstyle{same} % disable monospaced font for URLs
\hypersetup{
  pdftitle={Detecting computing-enabled interdisciplinary domains using the MIDFIELD data set},
  pdfauthor={Tim Ransom; Randi Sims; Stephanie Damas},
  pdfkeywords={IEEE, IEEEtran, journal, Quarto, Pandoc, template},
  colorlinks=true,
  linkcolor={blue},
  filecolor={Maroon},
  citecolor={Blue},
  urlcolor={Blue},
  pdfcreator={LaTeX via pandoc}}

% *** Do not adjust lengths that control margins, column widths, etc. ***
% *** Do not use packages that alter fonts (such as pslatex).         ***
% There should be no need to do such things with IEEEtran.cls V1.6 and later.
% (Unless specifically asked to do so by the journal or conference you plan
% to submit to, of course. )


% correct bad hyphenation here
\hyphenation{op-tical net-works semi-conduc-tor}

%
% paper title
% can use linebreaks \\ within to get better formatting as desired
% Do not put math or special symbols in the title.
% paper title
% can use linebreaks \\ within to get better formatting as desired
% Do not put math or special symbols in the title.
\title{Detecting computing-enabled interdisciplinary domains using the
MIDFIELD data set}

\author{
\thanks{The \texttt{quarto-ieee} template is freely available under the
MIT license on github: \url{https://github.com/dfolio/quarto-ieee}.}
Tim Ransom\orcidlink{0000-0003-0357-5427},~Randi
Sims\orcidlink{0000-0003-0357-5427}
and~Stephanie Damas\orcidlink{0000-0003-0357-5427}%
\thanks{Tim Ransom is with Engineering and Science Education, Clemson
University, Clemson, 29634 United States of America%
 e-mail: tsranso@clemson.edu}
%by-author.affiliations
\thanks{Randi Sims is with Engineering and Science Education, Clemson
University, Clemson, 29634 United States of America%
 e-mail: rsims@clemson.edu}
%by-author.affiliations
\thanks{Stephanie Damas is with Engineering and Science
Education, Clemson University, Clemson, 29634 United States of America%
 e-mail: damas@clemson.edu}
%by-author.affiliations
}
\begin{document}

% The paper headers
\markboth{Journal XXX, Month Year}{D. Folio: A Sample Article Using
quarto-ieee}

% use for special paper notices

% make the title area
\maketitle

% As a general rule, do not put math, special symbols or citations
% in the abstract or keywords.
\begin{abstract}
This document describes the most common article elements and how to use
the \texttt{quarto-ieee} class with Pandoc/Quarto-Markdown to produce
files that are suitable for submission to IEEE journals.
\texttt{quarto-ieee} can produce conference, journal, and technical note
(correspondence) papers with a suitable choice of class options. It
intends to generate PDF and HTML outputs that closely mimick what IEEE
would generate.
\end{abstract}
% Note that keywords are not normally used for peerreview papers.
\begin{IEEEkeywords}
IEEE, IEEEtran, journal, Quarto, Pandoc, template
\end{IEEEkeywords}

% For peer review papers, you can put extra information on the cover
% page as needed:
% \ifCLASSOPTIONpeerreview
% \begin{center} \bfseries EDICS Category: 3-BBND \end{center}
% \fi
%
% For peerreview papers, this IEEEtran command inserts a page break and
% creates the second title. It will be ignored for other modes.
% \IEEEpeerreviewmaketitle

\ifdefined\Shaded\renewenvironment{Shaded}{\begin{tcolorbox}[enhanced, sharp corners, frame hidden, borderline west={3pt}{0pt}{shadecolor}, boxrule=0pt, breakable, interior hidden]}{\end{tcolorbox}}\fi

\hypertarget{tims-draft}{%
\section{Tim's Draft}\label{tims-draft}}

A major contribution of the computing sciences historically has been to
enable other branches of science the knowledege and skills to further
their own endeavors with the use of computation.

This has led to the creation of many multidisciplinary fields such as
bioinformatics, computational chemistry, and computational physics.
These specialized fields offer both a targeted approach towards
furthering our collective knowledge as well as acting as a point of
attraction for steudents to enroll is one university over another.

The curriculum of these multidisciplinary programs often include courses
from their respective ``parent'' fields. For example, bioinformatics
curriculum can include courses from both computer science departments as
well as biology departments. The number of courses that are specific to
the ``child'' field need not be as large as the number of courses
offered by the ``parent'' field(s).

The MIDFIELD project provides a robust and detailed data set of the
courses that students have taken as they complete their degrees. This
project uses the MIDFIELD data set to determine the courses likely

\hypertarget{randi-draft}{%
\section{Randi Draft}\label{randi-draft}}

\hypertarget{problem-statement}{%
\subsection{Problem statement}\label{problem-statement}}

Degree programs for computer science students are often rigid, giving
students few choices in courses, course sequences, and time to
graduation. Courses within these degree programs have historically been
accessible only to computer science majors. In recent years {[}maybe{]},
universities have begun to shift particular computing courses to
intersect with those in other departments which rely on computational
methods, such as biology (bioinformatics) and genetics (genomics).

\hypertarget{importance-of-interdisciplinary}{%
\subsection{Importance of
interdisciplinary}\label{importance-of-interdisciplinary}}

Interdisciplinary degree programs such as these allow for a diversity of
entrance pathways for students from non-traditional backgrounds. These
differing entrance pathways can promote inclusivity, allowing students
multiple routes to graduation not outlined in a rigid degree pathway.
While the importance of interdisciplinary fields and programs are
well-known, the extent to which courses overlap between degree programs
has not been investigated.

\hypertarget{theoretical-framework}{%
\subsection{Theoretical Framework}\label{theoretical-framework}}

\hypertarget{gap-the-work-fills}{%
\subsection{Gap the work fills}\label{gap-the-work-fills}}

Our work seeks to better understand the flexibility of these courses and
their utilization within and between different degree programs through
the following research questions:

\hypertarget{rqs-and-hypotheses}{%
\subsection{RQs and hypotheses}\label{rqs-and-hypotheses}}

RQ1: What are the most overlapping degree programs in the computing
disciplines? RQ2: What are the most frequent courses to migrate between
within computing disciplines?

\hypertarget{methods}{%
\subsection{Methods}\label{methods}}

This project uses data from {[}x{]} institutions retrieved from the
MIDFIELD (Multi-Institutional Database for Investigating Engineering
Longitudinal Development). Using techniques from Boolean algebra and
descriptive statistics, we identified undergraduate degree programs with
the highest and lowest percentages of overlap with computer science
curriculums. We then used course data within the degree programs to
determine the levels of course overlap between degree programs.

\hypertarget{results}{%
\subsection{Results}\label{results}}

Our findings show {[}x{]} degree programs as having large (x)
percentages of overlap with computer science programs. {[}go into
courses that make up these programs and their overlap{]}

\hypertarget{discussions}{%
\subsection{Discussions}\label{discussions}}

\hypertarget{impacts}{%
\subsection{Impacts}\label{impacts}}

Results from this project indicate specific degrees and courses which
allow students easy entry points to an interdisciplinary degree pathway.


% Can use something like this to put references on a page
% by themselves when using endfloat and the captionsoff option.
\ifCLASSOPTIONcaptionsoff
  \newpage
\fi

% trigger a \newpage just before the given reference
% number - used to balance the columns on the last page
% adjust value as needed - may need to be readjusted if
% the document is modified later
%\IEEEtriggeratref{8}
% The "triggered" command can be changed if desired:
%\IEEEtriggercmd{\enlargethispage{-5in}}

% Uncomment when use biblatex with style=ieee
%\renewcommand{\bibfont}{\footnotesize} % for IEEE bibfont size

\pagebreak[3]
\begin{IEEEbiography}[\includegraphics{david-folio.png}]{Tim Ransom}
Tim Ransom is a PhD candidate in the Clemson University Enigineering and
Science Education department. He studies computer science education
through qualitative and computational methodologies.
\end{IEEEbiography}
\begin{IEEEbiography}[\includegraphics{david-folio.png}]{Randi Sims}
Tim Ransom is a PhD candidate in the Clemson University Enigineering and
Science Education department. He studies computer science education
through qualitative and computational methodologies.
\end{IEEEbiography}
\begin{IEEEbiography}[\includegraphics{david-folio.png}]{Stephanie
Damas}
Tim Ransom is a PhD candidate in the Clemson University Enigineering and
Science Education department. He studies computer science education
through qualitative and computational methodologies.
\end{IEEEbiography}
% that's all folks
\end{document}

